\documentclass[../thesis.tex]{subfiles}
\begin{document}

\section{Related work}

The work by Lowe \cite{lowe2004distinctive} revolutionized the discipline of detecting and describing features in images, and inspired many later descriptors. We will summarize the most successful of these. The descriptors are divided into two categories: spatial pooling and higher order descriptors. We also look at various performance evaluations of descriptors.

\subsection{Spatial pooling descriptors}

\emph{Spatial pooling} is the technique of combining spatially localized image information. The \emph{scale-invariant feature transform} (SIFT) descriptor \cite{lowe2004distinctive} is the most popular descriptor based on this approach. It works by dividing the image regions surrounding interest points into square cells and constructing a histogram of unsigned gradient orientations for the pixels in each cell. The interest points are detected using a multi-scale DoG detector. By computing the gradient orientations at detection scale, the descriptor becomes scale invariant. The descriptor is also rotation invariant as the cell grid is rotated according to the dominating gradient orientation. Finally the descriptor achieves illumination invariance by normalizing the histograms. The SIFT descriptor uses a grid of $4 \times 4$ cells each consisting of $4 \times 4$ pixels, with 8 histogram bins for each cell. This results in a dimensionality of $4 \times 4 \times 8 = 128$.

PCA-SIFT \cite{ke2004pca},
HOG \cite{dalal2005histograms},
DPM \cite{felzenszwalb2008discriminatively},
GLOH \cite{mikolajczyk2005performance},
DAISY \cite{tola2008fast,winder2009picking}

\subsection{Higher order differential descriptors}

Another approach to descriptor design is to use higher order differential information.

BIF-column texture representation \cite{crosier2010using},
$\mathcal{J}_4$-grid2 \cite{larsen2012jet},
Galaxy descriptor \cite{pedersen2013shape}

\subsection{Performance evaluations}

\cite{mikolajczyk2005performance},
\cite{dahl2011finding}

\subbibliography

\end{document}

