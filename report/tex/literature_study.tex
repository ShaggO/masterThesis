\documentclass[../thesis.tex]{subfiles}
\begin{document}

\section{Related work}

The work by Lowe \cite{lowe2004distinctive} revolutionized the discipline
of detecting and describing features in images, and inspired many later
descriptors. We will summarize the most successful of these. The descriptors
are divided into two categories: gradient-based and higher order descriptors.
We also look at various performance evaluations of descriptors.

\subsection{Gradient-based descriptors}
Gradients of an image are often used to describe features.
The \emph{scale-invariant feature transform} (SIFT) descriptor
\cite{lowe2004distinctive} is the most popular descriptor based on this
approach. It works by dividing the image regions surrounding interest
points into square cells and constructing a histogram of unsigned gradient
orientations for the pixels in each cell. The interest points are detected
using a multi-scale DoG detector. By computing the gradient orientations
at detection scale, the descriptor becomes scale invariant. The descriptor
is also rotation invariant as the cell grid is rotated according to the
dominating gradient orientation. Finally the descriptor achieves illumination
invariance by normalizing the histograms. The SIFT descriptor uses a grid of
$4 \times 4$ cells each consisting of $4 \times 4$ pixels, with 8 histogram
bins for each cell. This results in a dimensionality of $4 \times 4 \times 8 =
128$.

\emph{PCA-SIFT} \cite{ke2004pca} is also based on combining oriented gradients
within a region, but instead of dividing the feature into cell histograms
it uses principal component analysis (PCA) to reduce the dimensionality
of the combined gradients. It also uses a multi-scale DoG detector for
interest point detection as SIFT. When computing PCA we compute the most
significant orthogonal dimensions of a dataset. In order to obtain good
results a large and diverse dataset of images and their features is required.
PCA-SIFT describes a single patch of $41 \times 41$ pixels. When computing
the gradients it gives gradient dimensionality of $2\times39\times39 = 3042$,
which is reduced by PCA to 36 dimensions.

The \emph{gradient location and orientation histogram} (GLOH) descriptor
\cite{mikolajczyk2005performance} is an extension of the SIFT descriptor.
It differs from the SIFT descriptor by making a log-polar grid of
cells instead of rectangular cells. Furthermore the dimensionality of
the descriptor is reduced using PCA (on 47,000 patches according to
\cite{mikolajczyk2005performance}). The log-polar grid is split into three
rings at radius 6, 11, and 15, of which each of the two outer-most rings are
divided into eight angular cells giving a total of 17 cell bins. The gradient
orientations are divided into 16 bins giving a histogram of $16 \times 17
= 272$ bins. This is reduced to 128 dimensions by using the 128 largest
eigenvectors from PCA on a training dataset.

The \emph{DAISY} descriptor \cite{tola2008fast} is a descriptor originally
developed for dense wide-baseline matching, and it is therefore developed to
create a descriptor of each pixel in an image very efficiently. The descriptor
is created from Gaussian directional derivative convolutions of the image in
eight directions and at three different scales. The use of these convolutions
allows for fast computations of simple convolutions compared to the slower
binning and post-processing used in SIFT. The descriptor of each point is a
concatenation of the eight directional derivatives of 25 points located in a
circular grid.

HOG \cite{dalal2005histograms},
%DPM \cite{felzenszwalb2008discriminatively},
DAISY \cite{tola2008fast,winder2009picking}

\subsection{Higher order differential descriptors}

Another approach to descriptor design is to use higher order differential
information.

BIF-column texture representation \cite{crosier2010using},
$\mathcal{J}_4$-grid2 \cite{larsen2012jet},
Galaxy descriptor \cite{pedersen2013shape}

\subsection{Performance evaluations}

\cite{mikolajczyk2005performance},
\cite{dahl2011finding}

\subbibliography

\end{document}
