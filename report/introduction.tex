\documentclass[thesis.tex]{subfiles}
\begin{document}

\section{Introduction}
\label{sec:introduction}

Computer scientists have for many years tried to make computers interpret real world images as humans do, in order to solve problems such as image correspondence \cite{dahl2011finding}, pedestrian detection \cite{felzenszwalb2008discriminatively}, and content based image retrieval \cite{smeulders2000content}. The approaches to these problems are often based on creating descriptors to represent regions of the images, and comparing these descriptors by some method depending on the application. \todo{insert examples}

The strategy for choosing image regions depends on the application. If a systematic search across the image is needed, we use a sliding window approach. If we only want to describe the most distinctive regions, we instead use an interest point detector. The detectors that are popular and have proven to be successful are the Harris corner detector, Hessian based detectors, the Difference of Gaussian (DoG) blob detector, and the Maximally Stable Extremal Regions (MSER) detector \cite{aanaes2012interesting,dahl2011finding}.

An important part of descriptor design is to make them invariant to image transformations such as rotation, translation, illumination, scale, perspective, and noise. This causes real world objects to be described similarly despite being captured under different conditions. The choice of invariant properties depends on the application of the descriptor. For example rotation invariance is desired for texture description but not necessarily for recognition of pedestrians.

%%%% This is our proposed solution
% Should be changed to the final solution/what we have tried once we get that
% far in the process
We will try to design and implement a variant of the HOG descriptor utilizing higher order differential information.\mycomment[KSP]{Hvorfor er det interessant?} We will select a few of the following applications for testing our solution: Texture recognition, pedestrian detection, general object detection using deformable parts models, image correspondence, medical image registration, and content based image retrieval.

At first we will try to experiment with a HOG descriptor extended to various orders of differential information (like Jet descriptors à la \cite{larsen2012jet}) and various parameters for the number of cells per block, number of pixels per cell and number of bins per cell histogram. Furthermore we will experiment with the combination of HOG and the shape index \cite{koenderink1992surface} as well as different ways of processing the higher order information.
%%%% /proposed solution


%% Note on appendices in the report
In this report we try to focus on the essential parts of our study of higher order descriptors. In order to keep this focus we have chosen to omit certain derivation details from the report itself and put them in appendices instead. Throughout the report there are notes when details are moved to appendices.



\subbibliography

\end{document}
