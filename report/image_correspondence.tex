\documentclass[thesis.tex]{subfiles}
\begin{document}
\chapter{Image correspondence}

inconsistencies and estimation of 3D points

\section{Dataset}
The dataset we are using for training and evaluation of our descriptor is called the DTU Robot 3D dataset \cite{aanaes2010recall} (from now on called the DTU dataset). It concists of images of objects taken in a black box illuminated using 19 fixed positioned LED lights. In total there are 60 scenes with varying objects. \Cref{fig:dtu_examples} shows 4 example scenes. \citet{aanaes2010ground} classify the scenes as shown in \Cref{tbl:dtu_scene_classifications}.

\begin{figure}
	\centering
	\begin{subfigure}{0.4\textwidth}
		\includegraphics[width=\textwidth]{img/dtu_example_1.png}
	\end{subfigure}
	\begin{subfigure}{0.4\textwidth}
		\includegraphics[width=\textwidth]{img/dtu_example_2.png}
	\end{subfigure}
	\begin{subfigure}{0.4\textwidth}
		\includegraphics[width=\textwidth]{img/dtu_example_3.png}
	\end{subfigure}
	\begin{subfigure}{0.4\textwidth}
		\includegraphics[width=\textwidth]{img/dtu_example_4.png}
	\end{subfigure}
	\caption{Examples of scenes from the DTU dataset}
	\label{fig:dtu_examples}
\end{figure}

The camera is positioned around each scene using an industrial robot arm, which has automaticly captured each scene from 119 positions. These positions are defined from a fixed frontal view varying the viewpoint $\theta$ in three arcs at different distance $d$ to the scene. Arc 1 has $d = \SI{0.5}{\meter}$ to the scene and $\theta$ spans $\SI{\pm40}{\degree}$, arc 2 has $d = \SI{0.65}{\meter}$ and $\theta$ spans $\SI{\pm25}{\degree}$, and arc 3 has $d = \SI{0.8}{\meter}$ and $\theta$ spans $\SI{\pm20}{\degree}$. Furthermore a linear path is captured by moving the camera away from the scene, which corresponds to zooming or scaling the scene. This is done at $\theta = \SI{0}{\degree}$ and $d$ spans $[\SI{0.5}{\meter};\SI{0.8}{\meter} ]$. At each of the 119 camera positions 19 individual images $I_i$ are taken with each of the LED lights $i,~\text{for}~i = 1,\hdots,19$ turned on. Using the given camera positions we get four camera paths: three arc paths and one linear path.

The choice of capturing the scenes with individual LED lights created images with a high amount of casting shadows. \citet{larsen2012jet} created as set\footnote{dataset available at http://roboimagedata.imm.dtu.dk/data/condensed.tar.gz} of artificial diffuse and light paths from the individual LED images which we now briefly explain. The artificial diffuse light images are created from the individual lightings in order to only evaluate the performance of our descriptor under viewpoint changes and to get more natural images. These diffuse images are created by averaging over the individual light images for each camera position:
\begin{align}
	I_{\text{diffuse}} = \frac{1}{19} \sum_{i = 1}^{19} I_{i}
\end{align}
Since the dataset concist individual LED images, one is able to construct images simulating two light source paths going from right to left and back to front respectively.
Given a light position $L_{\boldsymbol{x}}$ in the spatial domain of the LED positions, the image $I_{\boldsymbol{x}}$ is constructed by weighting each LED image by the Gaussian of the distance to $L_{\boldsymbol{x}}$:
\begin{align}
	L_{x} = \sum_{i = 1}^{19} G(\boldsymbol{x} - \boldsymbol{x}_i,\sigma) I_{i}
\end{align}
In order to get a somewhat general measurement for the robustness against light variations there are light paths generated for the following four image positions: 12 (arc 1), 25 (arc 1), 60 (linear path), and 87 (arc 2).
See \citet{aanaes2010ground} for more information about the dataset and the generated light paths.


\Cref{fig:light_example} show 6 of the different light images for scene 4 at camera position 60. \Cref{fig:light_example_02,fig:light_example_17,fig:light_example_08} show the images taken with individual LED lighting (numbers 2, 17, and 8 respectively), and \Cref{fig:light_example_00} shows the diffuse light image. We here notice the significant difference in cast shadows using the three LEDs individually compared to the diffuse light image. The diffuse light image is however quite dark compared to the LED 8 light image, which could potentially cause problems in some scenes generating too few interest points. Such an issue could however be handled by adapting detection thresholds.
\Cref{fig:light_example_28,fig:light_example_20} show the left- and rightmost positions of the X light path. By comparing these to their LED counterparts (\subref{fig:light_example_02} and \subref{fig:light_example_17} respectively) we see that the cast shadows are less significant therefore looking more natural.
\Cref{fig:viewpoint_example} shows 4 different camera positions for scene 4 including the keyframe (position 25).
%
\begin{figure}
	\centering
	\begin{subfigure}{0.49\textwidth}
		\includegraphics[width=\textwidth]{img/scene_04_img60_02.png}
		\caption{LED 2 light}
		\label{fig:light_example_02}
	\end{subfigure}
	\begin{subfigure}{0.49\textwidth}
		\includegraphics[width=\textwidth]{img/scene_04_img60_17.png}
		\caption{LED 17 light}
		\label{fig:light_example_17}
		\end{subfigure}
	\begin{subfigure}{0.49\textwidth}
		\includegraphics[width=\textwidth]{img/scene_04_img60_08.png}
		\caption{LED 8 light}
		\label{fig:light_example_08}
	\end{subfigure}
	\begin{subfigure}{0.49\textwidth}
		\includegraphics[width=\textwidth]{img/scene_04_img60_00.png}
		\caption{Diffuse light}
		\label{fig:light_example_00}
	\end{subfigure}
	\begin{subfigure}{0.49\textwidth}
		\includegraphics[width=\textwidth]{img/scene_04_img60_28.png}
		\caption{Leftmost position of X light path}
		\label{fig:light_example_28}
	\end{subfigure}
	\begin{subfigure}{0.49\textwidth}
		\includegraphics[width=\textwidth]{img/scene_04_img60_20.png}
		\caption{Rightmost position of X light path}
		\label{fig:light_example_20}
	\end{subfigure}
	\caption{Examples of light images in scene 4 at camera position 60.}
	\label{fig:light_example}
\end{figure}
%
\begin{figure}
	\centering
	\begin{subfigure}{0.49\textwidth}
		\includegraphics[width=\textwidth]{img/scene_04_img12_00.png}
		\caption{Camera position 12 (arc 1)}
		\label{fig:viewpoint_example_left}
	\end{subfigure}
	\begin{subfigure}{0.49\textwidth}
		\includegraphics[width=\textwidth]{img/scene_04_img47_00.png}
		\caption{Camera position 47 (arc 1)}
		\label{fig:viewpoint_example_right}
	\end{subfigure}
	\begin{subfigure}{0.49\textwidth}
		\includegraphics[width=\textwidth]{img/scene_04_img25_00.png}
		\caption{Camera position 25 (keyframe)}
		\label{fig:viewpoint_example_keyframe}
	\end{subfigure}
	\begin{subfigure}{0.49\textwidth}
		\includegraphics[width=\textwidth]{img/scene_04_img64_00.png}
		\caption{Camera position 64 (linear path)}
		\label{fig:viewpoint_example_scale}
	\end{subfigure}
	\caption{Examples of various camera positions of scene 4 using diffuse light.}
	\label{fig:viewpoint_example}
\end{figure}
%
\begin{table}
	\centering
	\begin{tabular}{l l r}
		\toprule
		Class & Scene numbers & Total \\
		\midrule
		House					& 1, 4, 8, 31, 32, 49, 50, 55				& 8 \\
		Books					& 2, 11, 20, 21								& 4 \\
		Fabric					& 5, 6, 45, 46, 47, 48						& 6 \\
		Greens					& 23, 24, 25, 26, 27, 51, 52, 53, 54, 56	& 10 \\
		Beer  					& 15, 16									& 2 \\
		Teddy Bears 			& 9, 10, 43, 44								& 4 \\
		Building Materials 		& 33, 34, 35, 36, 37						& 5 \\
		Decorative Items (Art) 	& 38, 39, 40, 41, 42						& 5 \\
		Groceries 				& 12, 28, 29, 30							& 4 \\
		Twigs and Leaves 		& 17, 57, 58, 59, 60 						& 5 \\
		\bottomrule
	\end{tabular}
	\caption{Scene object classifications from \cite[Table 1]{aanaes2010ground}}
	\label{tbl:dtu_scene_classifications}
\end{table}
%

\subsection{Pitfalls and deficiencies}
The DTU dataset is built to test computer vision systems' ability to cope with viewpoint changes, scaling, light changes, and to a certain extend occlusions as well. Since all images are captured with the same camera tilt, no rotation of objects occurs and hence we adhere from persuing rotational invariance.

Compare with other datasets traditionally used: Good amount of viewpoints, not as natural images as other sets, no rotation. Well structured approach to changes in viewpoints and scale

\Cref{fig:dtu_problems} shows the problems we have found within the DTU dataset, which we will go through here.

The artificial diffuse light is, as mentioned in the previous section, created by combining images captured under individual LED lighting. This is not optimal as some of the images are visually suffering from the spatial layout of the LED lights. \Cref{fig:dtu_problems_diffuse} shows image 20 (arc 1) from set 2 of the dataset. The diffuse light problem is clearly visible in the cast shadows seen on the 1st and 3rd book in the stack of lying books. Ideally these shadows would form either a smoothed or a single hard shadow edge instead of a set of gradually fading hard edges.

The automatic capturing of images using a robot arm combined with the individual LED lighting has the side-effect of the robot arm casting shadows in some of the images. This is seen when using the 3rd LED light and having captured the scenes from the far left of arc 1. \Cref{fig:dtu_problems_robot} shows an example of a cast shadow originating from the robot arm. This is however only a problem with one of the 19 LED light images and hence the effect on the final test images is minimal.
%
\begin{figure}
	\centering
	\begin{subfigure}{0.49\textwidth}
		\includegraphics[width=\textwidth]{img/diffuse_light_problem.png}
		\caption{Diffuse lighting in set 2, image 20, with shadow artifacts}
		\label{fig:dtu_problems_diffuse}
	\end{subfigure}
	\begin{subfigure}{0.49\textwidth}
		\includegraphics[width=\textwidth]{img/robot_arm_shadow.png}
		\caption{Robot arm shadow in set 4, image 7, LED light 3}
		\label{fig:dtu_problems_robot}
	\end{subfigure}
	\caption{Examples of problematic images in the DTU dataset}
	\label{fig:dtu_problems}
\end{figure}
\section{Experiments}


\section{Parameter study}



Test: 10 fold cross validation

\section{Evaluation}

Confusion matrix
ROC AUC definition
PR AUC definition

\subbibliography

\end{document}
