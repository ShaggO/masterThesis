\documentclass[thesis.tex]{subfiles}
\begin{document}

\chapter{Discussion}

- Is it worth adding shape index?
- Importance of different parameters and design choices, relate these results to other articles
	- DTU
		- grid radius: 13.5 (area 572.5), SIFT: (area 256, 24x24 give 576)
		Grid size:
		We wish to compare our layout to that of GLOH and DAISY. Using our terminology, GLOH has size $8\times2$ with a central cell, and DAISY has size $8 \times 3$ with a central cell. Compared to these grids, we have more angular cells but the same amount of rings as GLOH. All the descriptors are however using a central cell.
		
		
		
- Optimal parameters are very different for the two applications, hard to choose a good overall descriptor. Biggest differences: smooth, kernels, norm. scale, final descriptor normalization
- Considered using multi-scale descriptors as the galaxy descriptor, but it doesn't seem like a good idea
- Dimensionality of descriptors, worth using compact GO?
- Running time: we have not focused on running time, optionally run a profiler, where can we save time? (don't use MATLAB)
- Overall results and comparison of descriptor design with SIFT/HOG. SIFT: not much difference, HOG: we beat them by using a dense descriptor with pixel normalization instead of block normalization

\section{Parameter study}


\section{Future work} % rename to alternative approaches?
- Improved pixel normalization:
	- normalize by the area which we integrated over to avoid boundary conditions
	- Some way of removing noise from dark areas
- PCA to reduce dimensionality of GO+SI?
- Two-way matching could improve matching results, but not important for comparing descriptors
- Object detection could be evaluated on harder datasets, as we have almost perfect classification on INRIA
- Higher level descriptors using our descriptor for better object classification
- Higher (than 2) order differential information
- Computing histograms at original scales, causing infeasible computation time. Maybe possible to limit the size and get good results?
- Distribution of shape index is not uniform, maybe changing bin locations could help. Reference "Shape index 2009"
- Better parameter optimization approaches. Simulated annealing?

As mentioned in \Cref{sec:icParameterStudy}, our scheme for optimizing parameters risks getting stuck in local optima. We defined our parameters to be as uncorrelated as possible, but it could still be the case that modifying several of the parameters together could give a better result. A possible way to improve our parameter optimization, while still being able to evaluate it in a reasonable time frame, is a simulated annealing approach

\section{Conclusion}

\subbibliography

\end{document}
