\documentclass[11pt,a4paper]{article}

\usepackage[utf8]{inputenc}
\usepackage[english]{babel}
\usepackage[T1]{fontenc}
\usepackage{lmodern}

\usepackage{amsmath,amssymb,amsfonts}

\title{\bfseries{Study of higher order image descriptors}\\Project description}
\author{
    Malte Stær Nissen \\\texttt{tgq958@alumni.ku.dk}
    \and
    Benjamin Michael Braithwaite \\ \texttt{cpg608@alumni.ku.dk}
    \\
    \\ \small{Supervisor: Kim Steenstrup Pedersen, \texttt{kimstp@diku.dk}}
    \\ \small{Co-supervisor: Sune Darkner, \texttt{darkner@diku.dk}}
    }

\begin{document}
\maketitle

\section{Problem statement}
Features/interest points are a central concept in computer vision. These
points mark areas of an image containing a significant amount of information
about the local structure. A visual descriptor is a representation of the
local structure of a feature. Such a descriptor can be modelled in numerous
ways depending on the application and desired properties. In this thesis we
will study state of the art visual descriptors and develop, implement, test,
and discuss our own descriptors based on this study.

\section{Motivation}
Computer scientists have for many years tried to make computers perceive the
world as humans do. This is done by performing analysis of real world images.
The analysis is often based on extraction of features in the images. These
features can then be be used for applications such as: object detection of
pedestrians \cite{felzenszwalb2008discriminatively},
content based image retrieval \cite{smeulders2000content},
feature matching for solving the image correspondence problem
\cite{dahl2011finding}, etc.
The state of the art methods are however
not perfect and hence we will try improve upon the accuracy of the methods.

\section{Proposed solution}
We will try to design and implement a variant of the HOG descriptor utilizing
higher order differential information. We
will select a few of the following applications for testing our solution:
Texture recognition, pedestrian detection, general object detection using
deformable parts models, image correspondence, medical image registration, and
content based image retrieval.

At first we will try to experiment with a HOG descriptor extended to various orders
of differential information and various parameters for the number of cells per
block, number of pixels per cell and number of channels per cell histogram.
Furthermore we will experiment with the combination of HOG and the shape index
\cite{koenderink1992surface}
as well as different ways of processing the higher order information.

\section{Learning goals}
The following learning goals will be the success criteria for our
project.
\begin{itemize}
    \item Perform a literature study of visual descriptors and their
        applications.
    \item Develop our own visual descriptor based on state of the art work
        within the field.
    \item Select and implement solutions of some of the mentioned applications
        of visual descriptors.
    \item Perform a parameter study of our solutions.
    \item Perform an empirical evaluation of our solutions on selected
        problems compared to state of the art solutions.
    \item Discuss the results of the empirical studies.
\end{itemize}

\section{Risk assessment}
The focus of the thesis is to gain insight into the field of descriptors and
develop our own non-trivial descriptor on the basis hereof. This makes the risk
of failure low since no requirements for the accuracy of the final product are
stated. There is however a risk of developing a solution which performs worse
than state of the art methods, which will however still have given us
significant insight into the field in question.

\section{Time schedule \small{(preliminary)}}

\begin{itemize}
    \item{\textbf{Literature study:}
        \textbf{Deadline:} Monday the 24th of February.}
    \item{\textbf{HOG and shape index:} Basic descriptor implementation and
        preliminary report section about HOG extended with shape index. \\
        \textbf{Deadline:} Wednesday the 5th of March.}
    \item{\textbf{Image correspondence implementation:} Feature matching
        implementation and test with descriptor on DTU robot dataset. Chosen to
        get a simple initial estimate of the viability of our descriptor choices.
        Report section about the subject written. \\
        \textbf{Deadline:} Wednesday the 19th of March.}
    \item{\textbf{Unknown workload:} Further work will be determined
    continously.}
    \item{\textbf{Final experiments/results:}
        \textbf{Deadline:} Sunday the 3rd of August}
    \item{\textbf{Final report:}
        \textbf{Deadline:} Sunday the 17th of August}
\end{itemize}

\bibliography{../bibliography}
\bibliographystyle{alpha}

\end{document}

